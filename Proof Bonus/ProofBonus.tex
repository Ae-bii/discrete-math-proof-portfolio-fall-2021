\documentclass[11pt]{article}
\usepackage[margin=0.6in]{geometry}
\usepackage{amsmath}  % a standard package for math writing
\usepackage{amssymb}
\usepackage{wrapfig}
\usepackage{listings}   % include the contents of code files (avoids version issues!)
%\usepackage{mathpazo} % a better font than the default, according to some
\usepackage{booktabs}  % for pretty tables
\usepackage[hidelinks]{hyperref}   % makes section links, bibliography links clickable
\usepackage{graphicx}   % include images
\usepackage{wrapfig}    % wrap text at the side of a smaller figure (don't bother doing this for your own figures)
\usepackage{bm}
\usepackage[dvipsnames]{xcolor}        % write text in pretty colors
\usepackage{multicol}
\usepackage{amsthm}
\newtheorem{theorem}{Theorem}
\newtheorem{definition}{Definition}

\title{Proof Portfolio Bonus}
\author{Anupam Bhakta}


\begin{document}
	
\maketitle

\begin{definition}[Divisible]
	An integer $n$ is {\bf divisible} be an integer $d$ is there exists an integer $k$ such that $n=d\times k$.
\end{definition}

\begin{theorem}
For all nonnegative integers $n$, 
\[2^{2n}-1\text{ is divisible by }3.\]
\end{theorem}

\begin{proof}
	We proceed by induction. Since $2^{2\cdot0}-1=0=3\cdot0$, the result holds for $n=0$. Assume that $2^{2k}-1$ is divisible by 3 for all nonnegative integers $k$. By the definition of divisibility, this means that $2^{2k}-1=3x \implies 2^{2k}=3x+1$ for some integer $x$. We show that $2^{2(k+1)}-1$ is divisible by 3 for all nonnegative integers. Observe that 
	\begin{align}
		2^{2(k+1)}-1 &= 2^{2}\cdot2^{2k}-1 \\
		&= 4(3x+1)-1 \\
		&= 12x+4-1 \\
		&= 3(4x-1),
	\end{align}
	where in (2), we use the Inductive Hypothesis. Since $x$ is an integer, $3(4x-1)$ is divisible by $3$. Thus, $2^{2n}-1$ is divisible by 3 for all nonnegative integers $n$.
\end{proof}

\end{document}