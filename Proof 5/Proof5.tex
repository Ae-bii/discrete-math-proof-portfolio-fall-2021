\documentclass[11pt]{article}
\usepackage[margin=0.6in]{geometry}
\usepackage{amsmath}  % a standard package for math writing
\usepackage{amssymb}
\usepackage{wrapfig}
\usepackage{listings}   % include the contents of code files (avoids version issues!)
\usepackage{mathpazo} % a better font than the default, according to some
\usepackage{booktabs}  % for pretty tables
\usepackage[hidelinks]{hyperref}   % makes section links, bibliography links clickable
\usepackage{graphicx}   % include images
\usepackage{wrapfig}    % wrap text at the side of a smaller figure (don't bother doing this for your own figures)
\usepackage{bm}
\usepackage[dvipsnames]{xcolor}        % write text in pretty colors
\usepackage{multicol}
\usepackage{amsthm}

\date{November 8, 2021}

\renewcommand\qedsymbol{$\blacksquare$}

\newtheorem{theorem}{Theorem}

\title{Proof Portfolio}
\author{Anupam Bhakta}


\begin{document}
	
\maketitle

\begin{theorem}
A graph G is regular if and only if $\overline{G}$ is regular.
\end{theorem}

\begin{proof}
Asumme a graph $G$ is $r$-regular with $n$ vertices. This means that every vertex $v\in V(\overline{G})$ is adjacent to $(n-1)-r$ vertices. As such, $\overline{G}$ is a $[(n-1)-r]$-regular graph. Thus, if $G$ is regular then $\overline{G}$ is also regular. \\

Assume a graph $\overline{G}$ is $r$-regular. We know that the complement of $\overline{G}$ is also regular due to the previous reasoning. By definition, the complement of a complement is the original object so $G$ is also regular. Thus, we can conclude a graph $G$ is regular if and only if $\overline{G}$ is regular.
\end{proof}

\end{document}