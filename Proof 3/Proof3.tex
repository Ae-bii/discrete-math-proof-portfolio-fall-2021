\documentclass[11pt]{article}
\usepackage[margin=0.6in]{geometry}
\usepackage{amsmath}  % a standard package for math writing
\usepackage{amssymb}
\usepackage{wrapfig}
\usepackage{listings}   % include the contents of code files (avoids version issues!)
\usepackage{mathpazo} % a better font than the default, according to some
\usepackage{booktabs}  % for pretty tables
\usepackage[hidelinks]{hyperref}   % makes section links, bibliography links clickable
\usepackage{graphicx}   % include images
\usepackage{wrapfig}    % wrap text at the side of a smaller figure (don't bother doing this for your own figures)
\usepackage{bm}
\usepackage[dvipsnames]{xcolor}        % write text in pretty colors
\usepackage{multicol}
\usepackage{amsthm}

\date{October 24, 2021}

\renewcommand\qedsymbol{$\blacksquare$}

\newtheorem{theorem}{Theorem}

\title{Proof Portfolio}
\author{Anupam Bhakta}


\begin{document}
	
\maketitle

\noindent The {\bf Fibonacci Sequence} $F_1, F_2, F_3, \dots$ is defined by 
\[F_n=\begin{cases}1& n=1,2\\F_{n-2}+F_{n-1}&n\ge 3\end{cases}\]

\begin{theorem}
Then $n$th Fibonacci Sequence is 
\[F_n=\frac{1}{\sqrt{5}}\left[\left(\frac{1+\sqrt{5}}{2}\right)^n-\left(\frac{1-\sqrt{5}}{2}\right)^n\right]\]
for every positive integer $n$.
\end{theorem}

\begin{proof}
	We proceed by the Strong Principle of Mathematical Induction. Since $F_1 = 1 = \frac{1}{\sqrt{5}}\left[\left(\frac{1+\sqrt{5}}{2}\right)^1-\left(\frac{1-\sqrt{5}}{2}\right)^1\right]$, the formula holds for n = 1. Assume, for a positive integer $k$, that $F_i = \frac{1}{\sqrt{5}}\left[\left(\frac{1+\sqrt{5}}{2}\right)^i-\left(\frac{1-\sqrt{5}}{2}\right)^i\right]$ for every $i$ with $1\le i\le k$. We show that $F_{k+1} = \frac{1}{\sqrt{5}}\left[\left(\frac{1+\sqrt{5}}{2}\right)^{k+1}-\left(\frac{1-\sqrt{5}}{2}\right)^{k+1}\right]$. First, observe that when $k=1$, $F_{k+1} = F_{1+1} = F_2 = 1 = \frac{1}{\sqrt{5}}\left[\left(\frac{1+\sqrt{5}}{2}\right)^2-\left(\frac{1-\sqrt{5}}{2}\right)^2\right]$ and so the formula holds. Hence we may assume that $k\ge2$. Since $k+1\ge3$, it follows by the recurrence relation that
	\begin{align}
		F_{k+1} &= F_{k-1}+F_k \\
		&= \frac{1}{\sqrt{5}}\left[\left(\frac{1+\sqrt{5}}{2}\right)^{k-1}-\left(\frac{1-\sqrt{5}}{2}\right)^{k-1}\right] + \frac{1}{\sqrt{5}}\left[\left(\frac{1+\sqrt{5}}{2}\right)^{k}-\left(\frac{1-\sqrt{5}}{2}\right)^{k}\right] \\
		&= \frac{1}{\sqrt{5}}\left[\left(\frac{1+\sqrt{5}}{2}\right)^{k-1}-\left(\frac{1-\sqrt{5}}{2}\right)^{k-1} + \left(\frac{1+\sqrt{5}}{2}\right)^{k}-\left(\frac{1-\sqrt{5}}{2}\right)^{k}\right] \\
		&= \frac{1}{\sqrt{5}}\left[\left(\frac{1+\sqrt{5}}{2}\right)^{k-1}\left(1+ \frac{1+\sqrt{5}}{2}\right) - \left(\frac{1+\sqrt{5}}{2}\right)^{k-1}\left(1+\frac{1-\sqrt{5}}{2}\right)\right] \\
		&= \frac{1}{\sqrt{5}}\left[\left(\frac{1+\sqrt{5}}{2}\right)^{k-1}\left(\frac{3+\sqrt{5}}{2}\right) - \left(\frac{1+\sqrt{5}}{2}\right)^{k-1}\left(\frac{3-\sqrt{5}}{2}\right)\right] \\
		&= \frac{1}{\sqrt{5}}\left[\left(\frac{1+\sqrt{5}}{2}\right)^{k-1}\left(\frac{1+\sqrt{5}}{2}\right)^2 - \left(\frac{1+\sqrt{5}}{2}\right)^{k-1}\left(\frac{1-\sqrt{5}}{2}\right)^2\right] \\
		&= \frac{1}{\sqrt{5}}\left[\left(\frac{1+\sqrt{5}}{2}\right)^{k+1}-\left(\frac{1-\sqrt{5}}{2}\right)^{k+1}\right],
	\end{align}
	where in (2), we use the Inductive hypothesis. \\
	It therefore follows by the Strong Principle of Mathematical Induction that $F_n=\frac{1}{\sqrt{5}}\left[\left(\frac{1+\sqrt{5}}{2}\right)^n-\left(\frac{1-\sqrt{5}}{2}\right)^n\right]$ for all positive integers $n$.
\end{proof}

\end{document}