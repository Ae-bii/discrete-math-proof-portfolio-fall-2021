\documentclass[11pt]{article}
\usepackage[margin=0.6in]{geometry}
\usepackage{amsmath}  % a standard package for math writing
\usepackage{amssymb}
\usepackage{wrapfig}
\usepackage{wasysym}
\usepackage{listings}   % include the contents of code files (avoids version issues!)
\usepackage{mathpazo} % a better font than the default, according to some
\usepackage{booktabs}  % for pretty tables
\usepackage[hidelinks]{hyperref}   % makes section links, bibliography links clickable
\usepackage{graphicx}   % include images
\usepackage{wrapfig}    % wrap text at the side of a smaller figure (don't bother doing this for your own figures)
\usepackage{bm}
\usepackage[dvipsnames]{xcolor}        % write text in pretty colors
\usepackage{multicol}
\usepackage{amsthm}
\newtheorem{theorem}{Theorem}
\newtheorem{definition}{Definition}
\AtBeginEnvironment{align}{\setcounter{equation}{0}}

\date{December 6, 2021}

\renewcommand\qedsymbol{$\smiley$}

\title{Proof Portfolio}
\author{Anupam Bhakta}


\begin{document}
	
\maketitle

\begin{theorem}
	Let $f:A\to B$ be a function and let $A_1$ and $A_2$ be subsets of $A$.  Prove that if $f$ is one-to-one then 
	\[f(A_1\cap A_2)=f(A_1)\cap f(A_2).\]
\end{theorem}

\begin{proof}
	Suppose $f:A\to B$ is an injective function. \\

	If $x \in f(A_1 \cap A_2)$, then there exists some $a \in A_1 \cap A_2$ such that $f(a) = x$. Since $a\in A_1\cap A_2$, then $a$ is also an element of $A_1$, which means $f(a)$ is an element of $f(A_1)$. In the same way, $f(a)$ is also an element of $f(A_2)$. As a result, $f(a)$ is an element of $f(A_1)\cap f(A_2)$ which implies $f(A_1\cap A_2)\subseteq f(A_1)\cap f(A_2)$. \\

	Next, let $y$ be an element of $f(A_1)\cap f(A_2)$. Then, there is an $a_1\in A_1$ and $a_2\in A_2$ such that $f(a_1)=y$ and $f(a_2)=y$. From this we can conclude that $a_1=a_2$ since $f(a_1)=f(a_2)$ and by the function being injective. This implies $f(A_1)\cap f(A_2)\subseteq f(A_1\cap A_2)$. \\

	The only way $f(A_1)\cap f(A_2)\subseteq f(A_1\cap A_2)$ and $f(A_1\cap A_2)\subseteq f(A_1)\cap f(A_2)$ is if $f(A_1\cap A_2)=f(A_1)\cap f(A_2)$.
\end{proof}

\newpage

\begin{theorem}
	Let $a$ be a fixed real number. Then 
	\[\sum_{i=0}^n(a+i)=\frac{1}{2}(n+1)(2a+n)\]
	for every nonnegative integer $n$.
\end{theorem}

\begin{proof}
	We proceed by induction. \\
	Let $n = 0$ be the smallest nonnegative integer. Then,
	$$\sum_{i=0}^{0}(a+i) = a = \frac{1}{2}(0+1)(2a+0).$$
	Thus, the result holds when $n = 0$. \\
	Assume that
	$$\sum_{i=0}^k(a+i)=\frac{1}{2}(k+1)(2a+k)$$
	for a nonnegative integer $k$. We show that $\sum_{i=0}^{k+1}(a+i)=\frac{1}{2}(k+2)(2a+k+1)$. Observe that
	\begin{align}
		\sum_{i=0}^{k+1}(a+i)&= (a+k+1) + \sum_{i=0}^{k}(a+i) \\
		&=(a+k+1) + \frac{1}{2}(k+1)(2a+k) \\
		&= \frac{1}{2}(2ak + 4a + k^2 +3k +2) \\
		&=\frac{1}{2}(k+2)(2a+k+1),
	\end{align}
	where in (2), we use the Inductive Hypothesis.\\\\
	Thus, by the Principle of Mathematical Induction, we conclude that $\sum_{i=0}^n(a+i)=\frac{1}{2}(n+1)(2a+n)$ for all nonnegative integers $n$.
\end{proof}

\newpage

\noindent The {\bf Fibonacci Sequence} $F_1, F_2, F_3, \dots$ is defined by 
\[F_n=\begin{cases}1& n=1,2\\F_{n-2}+F_{n-1}&n\ge 3\end{cases}\]

\begin{theorem}
Then $n$th Fibonacci Sequence is 
\[F_n=\frac{1}{\sqrt{5}}\left[\left(\frac{1+\sqrt{5}}{2}\right)^n-\left(\frac{1-\sqrt{5}}{2}\right)^n\right]\]
for every positive integer $n$.
\end{theorem}

\begin{proof}
	We proceed by the Strong Principle of Mathematical Induction. Since $F_1 = 1 = \frac{1}{\sqrt{5}}\left[\left(\frac{1+\sqrt{5}}{2}\right)^1-\left(\frac{1-\sqrt{5}}{2}\right)^1\right]$, the formula holds for n = 1. Assume, for a positive integer $k$, that $F_i = \frac{1}{\sqrt{5}}\left[\left(\frac{1+\sqrt{5}}{2}\right)^i-\left(\frac{1-\sqrt{5}}{2}\right)^i\right]$ for every $i$ with $1\le i\le k$. We show that $F_{k+1} = \frac{1}{\sqrt{5}}\left[\left(\frac{1+\sqrt{5}}{2}\right)^{k+1}-\left(\frac{1-\sqrt{5}}{2}\right)^{k+1}\right]$. First, observe that when $k=1$, $F_{k+1} = F_{1+1} = F_2 = 1 = \frac{1}{\sqrt{5}}\left[\left(\frac{1+\sqrt{5}}{2}\right)^2-\left(\frac{1-\sqrt{5}}{2}\right)^2\right]$ and so the formula holds. Hence we may assume that $k\ge2$. Since $k+1\ge3$, it follows by the recurrence relation that
	\begin{align}
		F_{k+1} &= F_{k-1}+F_k \\
		&= \frac{1}{\sqrt{5}}\left[\left(\frac{1+\sqrt{5}}{2}\right)^{k-1}-\left(\frac{1-\sqrt{5}}{2}\right)^{k-1}\right] + \frac{1}{\sqrt{5}}\left[\left(\frac{1+\sqrt{5}}{2}\right)^{k}-\left(\frac{1-\sqrt{5}}{2}\right)^{k}\right] \\
		&= \frac{1}{\sqrt{5}}\left[\left(\frac{1+\sqrt{5}}{2}\right)^{k-1}-\left(\frac{1-\sqrt{5}}{2}\right)^{k-1} + \left(\frac{1+\sqrt{5}}{2}\right)^{k}-\left(\frac{1-\sqrt{5}}{2}\right)^{k}\right] \\
		&= \frac{1}{\sqrt{5}}\left[\left(\frac{1+\sqrt{5}}{2}\right)^{k-1}\left(1+ \frac{1+\sqrt{5}}{2}\right) - \left(\frac{1+\sqrt{5}}{2}\right)^{k-1}\left(1+\frac{1-\sqrt{5}}{2}\right)\right] \\
		&= \frac{1}{\sqrt{5}}\left[\left(\frac{1+\sqrt{5}}{2}\right)^{k-1}\left(\frac{3+\sqrt{5}}{2}\right) - \left(\frac{1+\sqrt{5}}{2}\right)^{k-1}\left(\frac{3-\sqrt{5}}{2}\right)\right] \\
		&= \frac{1}{\sqrt{5}}\left[\left(\frac{1+\sqrt{5}}{2}\right)^{k-1}\left(\frac{1+\sqrt{5}}{2}\right)^2 - \left(\frac{1+\sqrt{5}}{2}\right)^{k-1}\left(\frac{1-\sqrt{5}}{2}\right)^2\right] \\
		&= \frac{1}{\sqrt{5}}\left[\left(\frac{1+\sqrt{5}}{2}\right)^{k+1}-\left(\frac{1-\sqrt{5}}{2}\right)^{k+1}\right],
	\end{align}
	where in (2), we use the Inductive hypothesis. \\
	It therefore follows by the Strong Principle of Mathematical Induction that $F_n=\frac{1}{\sqrt{5}}\left[\left(\frac{1+\sqrt{5}}{2}\right)^n-\left(\frac{1-\sqrt{5}}{2}\right)^n\right]$ for all positive integers $n$.
\end{proof}

\newpage

\begin{theorem}
	Let $A$ and $B$ be two sets and let $f:A\to B$ be a function.  
	If $|A| > |B|$ then $f$ is not injective (one-to-one).	
\end{theorem}

\begin{proof}
	We prove the contrapositive. Suppose that a function $f:A\to B$ is injective for two sets A and B. Since $f$ is injective, then different elements of $A$ must have different images in $B$. Therefore, if $|A| = n$, then the elements of $A$ have $n$ images in $B$. As a result, $|B|\ge n$. Thus, if $f: A\to B$ is injective, then $|A|\le|B|$.
\end{proof}

\newpage

\begin{theorem}
A graph G is regular if and only if $\overline{G}$ is regular.
\end{theorem}

\begin{proof}
Asumme a graph $G$ is $r$-regular with $n$ vertices. By definition, the complement of $G$ is $\overline{G}$ with $V(\overline{G})=V(G)$ such that two distinct vertices $u$ and $v$ of $G$ are adjacent in $\overline{G}$ if and only if $u$ and $v$ are not adjacent in G. This means that every vertex $v\in V(\overline{G})$ is adjacent to $(n-1)-r$ vertices. As such, $\overline{G}$ is a $[(n-1)-r]$-regular graph. Thus, if $G$ is regular then $\overline{G}$ is also regular. \\

Assume a graph $\overline{G}$ is $r$-regular. We know that the complement of $\overline{G}$ is also regular due to the previous reasoning. By definition, the complement of a complement is the original object so $G$ is also regular. Thus, we can conclude a graph $G$ is regular if and only if $\overline{G}$ is regular.
\end{proof}

\newpage

\begin{theorem}
A 3-regular graph G has a cut-vertex if and only if G has a bridge.
\end{theorem}

\begin{proof}
Assume $G$ is a 3-regular graph with a cut-vertex $v$. Then $G-v$ can be seperated into two components $G_1$ and $G_2$ or three components $G_1$, $G_2$, and $G_3$. Consider the case where there are two components. Since $G$ was originally a 3-regular graph, without loss of generality $v$ has a vertex $u$ that is adjacent in $G_1$ and two vertices that are adjacent in $G_2$. As such, there is a bridge $uv$ between the two components. Next, consider the case in which there are  three components. Then, $v$ has vertex $u$, $x$, and $y$ that is adjacent in $G_1$, $G_2$, and $G_3$ respectively. As such, there are bridges $uv$, $xv$, and $yv$.

Assume $G$ is a 3-regular graph with a bridge $uv$. Then the vertices of the bridge, $u$ and $v$, are also cut-vertices since removing them would remove their edges too. The bridge is a part of the edges that will be removed. Next, assume $G$ has bridges $uv$, $xv$, and $yv$. By the same logic used previously, the vertices of the bridges are also cut-vertices.

Thus we can conclude that a 3-regular graph $G$ has a cut-vertex if and only if $G$ has a bridge.
\end{proof}

\newpage

\begin{definition}[Divisible]
	An integer $n$ is {\bf divisible} be an integer $d$ if there exists an integer $k$ such that $n=d\times k$.
\end{definition}

\begin{theorem}
	For all nonnegative integers $n$, 
	\[2^{2n}-1\text{ is divisible by }3.\]
\end{theorem}

\begin{proof}
	We proceed by induction. Since $2^{2\cdot0}-1=0=3\cdot0$, the result holds for $n=0$. Assume that $2^{2k}-1$ is divisible by 3 for all nonnegative integers $k$. By the definition of divisibility, this means that $2^{2k}-1=3x \implies 2^{2k}=3x+1$ for some integer $x$. We show that $2^{2(k+1)}-1$ is divisible by 3 for all nonnegative integers. Observe that 
	\begin{align}
		2^{2(k+1)}-1 &= 2^{2}\cdot2^{2k}-1 \\
		&= 4(3x+1)-1 \\
		&= 12x+4-1 \\
		&= 3(4x-1),
	\end{align}
	where in (2), we use the Inductive Hypothesis. Since $x$ is an integer, $3(4x-1)$ is divisible by $3$. Thus, $2^{2n}-1$ is divisible by 3 for all nonnegative integers $n$.
\end{proof}

\end{document}