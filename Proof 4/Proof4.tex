\documentclass[11pt]{article}
\usepackage[margin=0.6in]{geometry}
\usepackage{amsmath}  % a standard package for math writing
\usepackage{amssymb}
\usepackage{wrapfig}
\usepackage{listings}   % include the contents of code files (avoids version issues!)
\usepackage{mathpazo} % a better font than the default, according to some
\usepackage{booktabs}  % for pretty tables
\usepackage[hidelinks]{hyperref}   % makes section links, bibliography links clickable
\usepackage{graphicx}   % include images
\usepackage{wrapfig}    % wrap text at the side of a smaller figure (don't bother doing this for your own figures)
\usepackage{bm}
\usepackage[dvipsnames]{xcolor}        % write text in pretty colors
\usepackage{multicol}
\usepackage{amsthm}
\newtheorem{theorem}{Theorem}

\date{October 31, 2021}

\renewcommand\qedsymbol{$\blacksquare$}

\title{Proof Portfolio}
\author{Anupam Bhakta}


\begin{document}
	
\maketitle

\begin{theorem}
	Let $A$ and $B$ be two sets and let $f:A\to B$ be a function.  
	If $|A| > |B|$ then $f$ is not injective (one-to-one).	
\end{theorem}

\begin{proof}
	We prove the contrapositive. For two nonempty finite set A and B, suppose that $f:A\to B$ is injective. Then different elements of $A$ must have different images in $B$. Therefore, if $A$ has $n$ elements, then the elements of $A$ have $n$ images in $B$. Consequently, the set $B$ must contain at least $n$ elements. Therefore, if $f: A\to B$ is injective, then $|A|\le|B|$.
\end{proof}

\end{document}